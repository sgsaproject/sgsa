\documentclass[12pt,a4paper]{article}
\usepackage[a4paper, inner=1.5cm, outer=2cm, top=3cm, bottom=2cm, bindingoffset=1cm]{geometry}
\usepackage[utf8]{inputenc}
\usepackage[english,brazilian]{babel}
\usepackage{graphicx}
\usepackage{tabularx,calc}
\usepackage{placeins}
\usepackage{color}
\usepackage{xcolor,colortbl}


\newcounter{Number}
\newcommand{\Item}{\stepcounter{Number}\theNumber}
\newcounter{NumberReqF}
\newcommand{\ReqF}{F\stepcounter{NumberReqF}\theNumberReqF ~- }
\newcounter{NumberReqNF}[NumberReqF]
\newcommand{\ReqNF}{NF \stepcounter{NumberReqNF}\theNumberReqF .\theNumberReqNF ~- }


\newcolumntype{Y}[1]{>{\setlength\hsize{#1\hsize}%
    \raggedright\arraybackslash}X}
\newcolumntype{F}{|p{\columnwidth-2\tabcolsep-2\arrayrulewidth}|}
\setlength\extrarowheight{2pt}

\newcommand\NFReq[5]{%
  \ReqNF #1 & #2 & #3 & (#4) & (#5)\tabularnewline
  \hline
}

\newcommand{\TabelaRequisito}[4]{
  \setcounter{NumberReqNF}{0}
  \begin{table}[h]
    \noindent\begin{tabularx}{\columnwidth}{|Y{0.4}|Y{0.5}|Y{0.25}|c|c|}
      \hline
      \multicolumn{3}{|l|}{\ReqF #1} & \multicolumn{2}{l|}{Hidden (#2)}\tabularnewline
      \hline
      \multicolumn{5}{F}{#3}\tabularnewline
      \hline
      \multicolumn{5}{|c|}{Non-Functional Requirements}\tabularnewline
      \hline
      Name & Restriction & Category & Desirable & Permanent\tabularnewline
      \hline
      #4%
      % more and more requirements here...
    \end{tabularx}
  \end{table}
}



   % Funções

\begin{document}

    \title{
    	Especificação de Requisitos \\ 
    	Sistema de Gerenciamento de Semana Acadêmica \\
    	SGSA
    }
    \author{Rafael Tavares Amorim \\ Thiago Krug}
    \date{22/01/2012}
    \maketitle

	\newpage
    
    \tableofcontents
    
    \newpage
    \section{Introdução}
    	\subsection{Propósito deste documento}
        \subsection{Escopo do produto}
        	\subsubsection{Nome do produto e de seus componentes principais}
        	\subsubsection{Missão do produto}
        	\subsubsection{Limite do produto}
        	\subsubsection{Benefícios do produto}
        	\begin{table}[h]
	        	\noindent\begin{tabular}{|p{0.30\columnwidth}|p{0.30\columnwidth}|p{0.40\columnwidth}|}
	        		\hline
	        		\setcounter{Number}{0}
	        	    \textbf{Número de ordem} & \textbf{Benefício} & \textbf{Valor para o cliente} \\
	        		\hline
	        		\Item & Controle do Evento & Alto \\
	        		\hline
	        		\Item & Inscrição dos Participantes & Médio \\
	        		\hline
	        		\Item & Gerar Credencial & Médio \\
	        		\hline
	        		\Item & Gerar Certificado & Baixo \\
	        		\hline
	        	\end{tabular}
        	\end{table}
        	
        	
        	\subsubsection{Definições, acrônimos e abreviações}
        	        	\begin{table}[h]
        		        	\noindent\begin{tabular}{|p{0.30\columnwidth}|p{0.30\columnwidth}|p{0.40\columnwidth}|}
        		        		\hline
        		        		\setcounter{Number}{0}
        		        		\textbf{Número de ordem} & \textbf{Termo} & \textbf{Definição} \\
        		        		\hline
        		        		\Item & LOL & LOL \\
        		        		\hline
        		        		\Item & LOL & LOL \\
        		        		\hline
        		        		\Item & LOL & LOL \\
        		        		\hline
        		        	\end{tabular}
        	        	\end{table}
        	        	
        	\subsubsection{Referências}
        	        	\begin{table}[!h]
        		        	\noindent\begin{tabular}{|p{0.30\columnwidth}|p{0.30\columnwidth}|p{0.40\columnwidth}|}
        		        		\hline
        		        		\setcounter{Number}{0}
        		        		\textbf{Número de ordem} & \textbf{Tipo Material} & \textbf{Referência bibliográfica} \\
        		        		\hline
        		        		\Item & LOL & LOL \\
        		        		\hline
        		        		\Item & LOL & LOL \\
        		        		\hline
        		        		\Item & LOL & LOL \\
        		        		\hline
        		        	\end{tabular}
        	        	\end{table}	
        \newpage	
        \section{Visão Geral}
        
        \newpage	
        \section{Requisitos Funcionais e Não-Funcionais Associados}
        
        	\subsection{Requisitos funcionais e requisitos não-funcionais associados}
        

		\setcounter{NumberReqF}{0}
		\setcounter{NumberReqNF}{0}
        	
        	\TabelaRequisito{Realizar Inscrição}{}
        	{Permite a inscrição do participante pelo site}
        	{%
        	  \NFReq{Campos}
        	  {Devem ser informados nome, data nascimento, instuição, campus, email e senha}
        	  {Especificação}{X}{}%
        	}
        	
        	\TabelaRequisito{Solicitar Isenção}{}
        	{Permite o participante solicitar isenção de pagamento no evento}
        	{%
        	}
        	
        	
        	
        	\TabelaRequisito{Realizar Login}{}
        	{O ouvinte coloca sua identificação e acessa a sua conta.}
        	{%
        	
        	}
        	        	        	
        	\FloatBarrier	
        	
        	
        	\subsection{Requisitos Suplementares}
        	
        	\subsection{Requisitos Prioritários}
        	
        	\newpage
        	\section{Requisitos Organizados}
	        	\subsection{Casos de Uso}
	        	
	        		\begin{tabular}{|l|l|l|l|}
	        		\hline Nome & Atores & Descrição & Ref. Cruzadas \\ 
	        		\hline  &  &  &  \\ 
	        		\hline 
	        		\end{tabular} 
	        		
	        	\subsection{Diagrama de Casos de Uso}
	        	\subsection{Conceitos}
	        	\subsection{Consultas}
	        	\subsection{Casos de Uso Expandidos}
        	
        	\newpage
        	\section{Modelo Conceitual}
        	
        	
        	
\end{document}
