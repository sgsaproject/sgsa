\documentclass[12pt,a4paper]{article}
\usepackage[a4paper, inner=1.5cm, outer=2cm, top=3cm, bottom=2cm, bindingoffset=1cm]{geometry}
\usepackage[utf8]{inputenc}
\usepackage[english,brazilian]{babel}
\usepackage{graphicx}
\usepackage{tabularx,calc}
\usepackage{placeins}
\usepackage{color}
\usepackage{xcolor,colortbl}
\usepackage{morefloats}
\usepackage{setspace}

\newcounter{Number}
\newcommand{\Item}{\stepcounter{Number}\theNumber}
\newcounter{NumberReqF}
\newcommand{\ReqF}{F\stepcounter{NumberReqF}\theNumberReqF ~- }
\newcounter{NumberReqNF}[NumberReqF]
\newcommand{\ReqNF}{NF \stepcounter{NumberReqNF}\theNumberReqF .\theNumberReqNF ~- }


\newcolumntype{Y}[1]{>{\setlength\hsize{#1\hsize}%
    \raggedright\arraybackslash}X}
\newcolumntype{F}{|p{\columnwidth-2\tabcolsep-2\arrayrulewidth}|}
\setlength\extrarowheight{2pt}

\newcommand\NFReq[5]{%
  \ReqNF #1 & #2 & #3 & (#4) & (#5)\tabularnewline
  \hline
}

\newcommand{\TabelaRequisito}[4]{
  \setcounter{NumberReqNF}{0}
  \begin{table}[h]
    \noindent\begin{tabularx}{\columnwidth}{|Y{0.4}|Y{0.5}|Y{0.25}|c|c|}
      \hline
      \multicolumn{3}{|l|}{\ReqF #1} & \multicolumn{2}{l|}{Hidden (#2)}\tabularnewline
      \hline
      \multicolumn{5}{F}{#3}\tabularnewline
      \hline
      \multicolumn{5}{|c|}{Non-Functional Requirements}\tabularnewline
      \hline
      Name & Restriction & Category & Desirable & Permanent\tabularnewline
      \hline
      #4%
      % more and more requirements here...
    \end{tabularx}
  \end{table}
}



   % Funções

\begin{document}

    \title{
    	Especificação de Requisitos \\ 
    	Sistema de Gerenciamento de Semana Acadêmica \\
    	SGSA
    }
    \author{Rafael Tavares Amorim \\ Thiago Krug}
    \date{22/01/2012}
    \maketitle

	%\newpage
	\clearpage
    
    \tableofcontents
    
    %\newpage
    \clearpage
    \section{Introdução}
    	\subsection{Propósito deste documento}
        \subsection{Escopo do produto}
        	\subsubsection{Nome do produto e de seus componentes principais}
        	\subsubsection{Missão do produto}
        	\subsubsection{Limite do produto}
        	\subsubsection{Benefícios do produto}
        	\begin{table}[h]
	        	\noindent\begin{tabular}{|p{0.30\columnwidth}|p{0.30\columnwidth}|p{0.32\columnwidth}|}
	        		\hline
	        		\setcounter{Number}{0}
	        	    \textbf{Número de ordem} & \textbf{Benefício} & \textbf{Valor para o cliente} \\
	        		\hline
	        		\Item & Controle do Evento & Alto \\
	        		\hline
	        		\Item & Inscrição dos Participantes & Médio \\
	        		\hline
	        		\Item & Gerar Credencial & Médio \\
	        		\hline
	        		\Item & Gerar Certificado & Baixo \\
	        		\hline
	        	\end{tabular}
        	\end{table}
        	
        	
        	\subsubsection{Definições, acrônimos e abreviações}
        	        	\begin{table}[h]
        		        	\noindent\begin{tabular}{|p{0.30\columnwidth}|p{0.30\columnwidth}|p{0.32\columnwidth}|}
        		        		\hline
        		        		\setcounter{Number}{0}
        		        		\textbf{Número de ordem} & \textbf{Termo} & \textbf{Definição} \\
        		        		\hline
        		        		\Item & LOL & LOL \\
        		        		\hline
        		        		\Item & LOL & LOL \\
        		        		\hline
        		        		\Item & LOL & LOL \\
        		        		\hline
        		        	\end{tabular}
        	        	\end{table}
        	        	
        	\subsubsection{Referências}
        	        	\begin{table}[!h]
        		        	\noindent\begin{tabular}{|p{0.30\columnwidth}|p{0.30\columnwidth}|p{0.32\columnwidth}|}
        		        		\hline
        		        		\setcounter{Number}{0}
        		        		\textbf{Número de ordem} & \textbf{Tipo Material} & \textbf{Referência bibliográfica} \\
        		        		\hline
        		        		\Item & LOL & LOL \\
        		        		\hline
        		        		\Item & LOL & LOL \\
        		        		\hline
        		        		\Item & LOL & LOL \\
        		        		\hline
        		        	\end{tabular}
        	        	\end{table}	
        %\newpage
        \clearpage
        \section{Visão Geral do Sistema}
        
        \begin{onehalfspace}
		A Semana Acadêmica do Centro de Tecnologia de Alegrete (SACTA) é um estímulo aos alunos à troca de conhecimentos nos âmbitos acadêmico e profissional, à exposição de suas pesquisas e uma oportunidade de aumentar seus conhecimentos.
        
        Durante a semana acadêmica acontecem palestras e mini-cursos onde os participantes podem aprender com outros alunos, com professores ou profissionais atuantes no mercado sobre assuntos de interesse a cada um dos cursos do Campus.
        
        A Semana Acadêmica do Centro de Tecnologia de Alegrete é organizada pelo Centro Estudantil, com apoio dos professores e alunos do campus bem como de empresas patrocinadoras.
        
        O que?
        
        O SGSA é um sistema para a gestão da Semana Acadêmica.
        
        Para que?
        
        Para realizar a gestão da Semana Acadêmica desde a inscrição, pagamento, geração de credenciais, controle de ouvintes até a finalização do evento, contabilizando as participações dos ouvintes.
        
        Para quem?
        
        Toda comunidade acadêmica (discentes, docentes, técnico-administrativos) e comunidade externa.
        
        Por que?
        \begin{itemize}
        \item Facilitar o controle e agilizar tanto para organização quanto para os participantes.
        \item Também para fornecer informações sobre o evento para todos.
        \item Tornar mais rápida a entrada dos participantes nas palestras/mini-cursos
        \item Tornar mais rápida a inscrição dos participantes no evento.
        \item Controlar a entrada e saída dos participantes.
        \item Controlar as palestras.
        \item Controlar as salas e locais do evento.
        \item Gerar relatórios.
        \item Certificado de participação
        \item Participantes por palestra.
        \item Palestras (dados das palestras)
        \item Participantes (dados dos participantes)
        \end{itemize}
        
        Quando?
        
        A semana acadêmica deste ano de 2012 ocorrerá de 11 a 15 de junho. O sistema precisará ficar pronto, ou pelo menos com uma versão beta até 15 dias antes do início do evento, ou seja ate 31/05/2012.
        
        Onde?
        
        UNIPAMPA Campus Alegrete, nas salas de aula e laboratórios disponíveis. Pode haver palestras/mini-cursos fora do prédio. Também podem existir visitações em outros lugares da cidade de Alegrete. Assim o sistema não pode restringir que os eventos possam ocorrer em ambientes externos, sem conexão a internet.
        
        Como?
        
        Inicialmente o sistema será aberto as inscrições dos interessados, disponibilizado pela internet. Logo após, haverá a fase de credenciamento onde a organização irá receber o pagamento, marcar como pago e gerar a credencial do participante e o recibo de pagamento, impresso na hora. Durante o evento, antes de iniciar a palestra haverá um representante da organização no local da mesma para fazer a entrada dos participantes. O participante, munido de sua credencial, entrega ao representante, no qual registra a entrada do participante na palestra, e devolve sua credencial. Por fim, é gerado o relatório.
        \end{onehalfspace}
        %\newpage
        \clearpage
        \section{Requisitos Funcionais e Não-Funcionais Associados}
        
        	\subsection{Requisitos funcionais e requisitos não-funcionais associados}
			\setcounter{NumberReqF}{0}
			\setcounter{NumberReqNF}{0}
			\subsubsection{Area do Site}
        	
				\TabelaRequisito{Realizar inscrição}{}
				{Permite a realização inscrição do participante pelo site}
				{%
					  \NFReq{Campos}
					  {Devem ser informados nome, data nascimento, RG, CPF, instituição, campus, curso, e-mail e senha}
					  {Especificação}{}{X}%
				}
				
				\TabelaRequisito{Solicitar isenção}{}
				{Permite o participante solicitar isenção de pagamento no evento}
				{%
					\NFReq{Campos}
					  {Deve ser informado o motivo}
					  {Especificação}{}{X}%
					\NFReq{Documento}
					  {Permitir o envio de arquivos}
					  {Especificação}{}{X}%
				}
				
				\TabelaRequisito{Realizar login}{}
				{O participante coloca sua identificação e acessa a sua conta.}
				{%
					\NFReq{Campos}
					  {O participante deve informar seu email e senha}
					  {Especificação}{}{}%
					\NFReq{Verificação de email}
					  {O participante somente poderá ter acesso após confirmar seu endereço de email}
					  {Especificação}{}{}%
				}
				
										
				\TabelaRequisito{Recuperar senha}{}
				{O participante solicita a recuperação de senha e a nova senha temporária é enviada por email}
				{%
					\NFReq{Campos}
					  {O participante deve informar seu email}
					  {Especificação}{}{}%
				}
				
				\TabelaRequisito{Mostrar timeline das palestras}{}
				{Mostra numa linha do tempo as palestras que estão ocorrendo e irão ocorrer.}
				{%
				
				}
				
				\TabelaRequisito{Mostrar programação do evento}{}
				{Mostra o calendário de programação do evento}
				{%
				
				}
				
				\TabelaRequisito{Enviar mensagem de contato}{}
				{Envia uma mensagem de contato para organização}
				{%
				
				}
				
				\TabelaRequisito{Validar certificado}{}
				{Verificar a validade do certificado}
				{%
				
				}
			
			\clearpage
			\subsubsection{Area do Participante}
			
				\TabelaRequisito{Editar dados do participante}{}
				{O participante acessa sua conta e edita seu proprios dados}
				{%
					\NFReq{Dados editaveis}
					  {O participante pode editar todos os dados com exceção do email que deve ser confirmado novamente}
					  {Especificação}{}{}%
				
				}
				
				\TabelaRequisito{Visualizar suas participações}{}
				{O participante pode acessar o site e ver suas palestras em que participou.}
				{%
					\NFReq{Dados disponiveis}
						{O participante poderá visualizar o nome, horario de inicio e fim, horario de entrada e saida, quantidade de hora e total de horas}
						{Especificação}{}{}%
				}
			
			\clearpage
			\subsubsection{Area do Administrador}
			
			\TabelaRequisito{Realizar pagamento}{}
        	{Marca o pagamento do participante}
        	{%
        	
        	}
			
			\TabelaRequisito{Emitir recibo de pagamento}{}
        	{Emite e imprime o recibo de pagamento do participante}
        	{%
        	
        	}
			
			\TabelaRequisito{Gerar credencial}{}
        	{Gera a credencial de um ou mais participantes}
        	{%
        	
        	}
			
			\TabelaRequisito{Adicionar ouvinte em palestra}{}
        	{Adiciona um participante como ouvinte em uma palestra mediante a apresentação da credencial}
        	{%
        	
        	}
			\TabelaRequisito{Fechar ouvinte em palestra}{}
        	{Finaliza a participação do ouvinte em uma palestra}
        	{%
        	
        	}
			
			\TabelaRequisito{Adicionar ouvinte em palestra após ter ocorrido a palestra}{}
        	{Adiciona um participante como ouvinte em uma palestra mediante a apresentação da credencial}
        	{%
        	
        	}
			
			\TabelaRequisito{Fechar ouvinte em palestra após ter ocorrido a palestra}{}
        	{Finaliza a participação do ouvinte em uma palestra}
        	{%
        	
        	}
			
			\TabelaRequisito{Notificar participante em outra palestra}{}
        	{Notifica sobre um participante que está em outra palestra e oferece possibilidade de fechar a participação}
        	{%
        	
        	}
			
			\TabelaRequisito{Editar palestras do ouvinte}{}
        	{Edita as palestras em que o ouvinte esteve, alterando as horas de entrada e saída. Pode também incluí-lo em uma ou finalizar sua participação numa palestra (ver os dois itens acima). Também pode excluir sua participação em uma palestra.}
        	{%
        	
        	}
						
			
			\TabelaRequisito{Importar ouvintes em palestra}{}
        	{Importa os ouvintes em uma palestra que tenha ocorrido em local onde não estava disponível acesso à internet, por exemplo.}
        	{%
        	
        	}
			
			\TabelaRequisito{Iniciar palestra}{}
        	{Inicia o contador de tempo de uma palestra, zera o contador dos ouvintes e marca palestra como iniciada}
        	{%
        	
        	}
			
			\TabelaRequisito{Finalizar palestra}{}
        	{Finaliza o contador da palestra e dos ouvintes. Marca a palestra como finalizada.}
        	{%
        	
        	}
			
			\TabelaRequisito{Alterar status da palestra}{}
        	{A palestra pode ser cancelada e outras coisas...}
        	{%
        	
        	}
			
			\TabelaRequisito{Cadastro de local}{}
        	{Permite o cadastramento de locais para o acontecimento das palestras}
        	{%
        	
        	}
			
			\TabelaRequisito{Cadastrar instituição de ensino}{}
        	{Cadastra a instituição de ensino onde ocorrerá o evento}
        	{%
        	
        	}
					
			
			\TabelaRequisito{Cadastrar campus}{}
        	{Cadastra o campus que ocorrerá o evento}
        	{%
        	
        	}
			
			\TabelaRequisito{Cadastrar evento}{}
        	{Cadastra o evento a ser realizado pela instituição }
        	{%
        	
        	}
			
			\TabelaRequisito{Cadastrar organização do evento}{}
        	{Cadastra a organização do evento}
        	{%
        	
        	}
			
			\TabelaRequisito{Cadastrar patrocinador}{}
        	{Cadastra um patrocinador do evento}
        	{%
        	
        	}
			
			\TabelaRequisito{Alterar dados básico do sistema (servidor email, etc)}{}
        	{}
        	{%
        	
        	}
			
			\TabelaRequisito{Gerenciar permissões/grupos}{}
        	{Gerencia as permissões de acesso de cada grupo de usuários}
        	{%
        	
        	}
			
			\TabelaRequisito{Gerenciar pessoa (adicionar, editar, remover, inativar/ativar)}{}
        	{Gerencia todo os usuários cadastrados no sistema}
        	{%
        	
        	}
			
			\TabelaRequisito{Gerenciar palestra (adicionar, editar, remover)}{}
        	{Gerencia as palestras no sistema}
        	{%
        	
        	}
			
			
			
			\TabelaRequisito{Enviar email de ativação de conta}{}
        	{Envia um email para que o participante confirme sua conta}
        	{%
        	
        	}
			
			\TabelaRequisito{Enviar email de confirmação de conta}{}
        	{Envia um email informando que a conta do participante foi confirmada}
        	{%
        	
        	}

			\TabelaRequisito{Enviar email de confirmação de inscrição em palestra}{}
        	{Envia um email confirmando a inscrição de um participante numa palestra}
        	{%
        	
        	}
			
			\TabelaRequisito{Realizar ativação da conta}{}
        	{Realiza a ativação da conta do participante por meio do link enviado via email}
        	{%
        	
        	}

			\TabelaRequisito{Gerar relatório}{}
        	{Gera relatórios sobre o evento}
        	{%
        	
        	}

        	\clearpage
        	\subsection{Requisitos Suplementares}
        	
        	\subsection{Requisitos Prioritários}
        	
        	\clearpage
        	\section{Requisitos Organizados}
	        	\subsection{Casos de Uso}
	        	
	        		\begin{tabular}{|l|l|l|l|}
	        		\hline Nome & Atores & Descrição & Ref. Cruzadas \\ 
	        		\hline  &  &  &  \\ 
	        		\hline 
	        		\end{tabular} 
	        		
	        	\subsection{Diagrama de Casos de Uso}
	        	\subsection{Conceitos}
	        	\subsection{Consultas}
	        	\subsection{Casos de Uso Expandidos}
        	
        	\clearpage
        	\section{Modelo Conceitual}
        	
        	        	
\end{document}
